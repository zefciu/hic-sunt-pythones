\abstract{
	Chcąc przedstawić Pythona jako programowalny język programowania omawiam dwa
	zagadnienia: deskryptory i metaklasy. Deskryptory są specjalnymi obiektami
	należącymi do klasy i zmieniającymi sposób dostępu do członków jej
	instancji. Są one wszechobecne w Pythonie i korzystamy z nich nawet o tym
	nie wiedząc. Metaklasy to ,,klasy klas''. Dzięki nim możemy dodać własną
	logikę do procesu tworzenia klas i dziedziczenia. Wykorzystanie tych
	elementów języka pozwoli nam tworzyć obiekty o niedostępnych normalnie
	możliwościach, jednocześnie proste i intuincyjne w obsłudze.
}

Czasami w ramach PR-u różnych trudnych i zawiłych języków programowania słyszymy
pewne przeciwstawienie ,,programowalnych języków programowania'' językom
prostym, ale ,,sztywnym'', nie dającym programiście swobody. Kiedy czynimy
pierwsze kroki w Pythonie, może nam się wydawać, że zalicza się on do tej
drugiej grupy. Nic bardziej mylnego. Pod warstwą prostoty umożliwiającą każdemu
szybkie wdrożenie znajduje się skomplikowany mechanizm, w którym praktycznie
wszystko można zmienić i zastąpić. 

\section{Descryptory}
\section{Metaklasy}
